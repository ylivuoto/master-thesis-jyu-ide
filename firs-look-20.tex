% Created 2024-03-23 la 16:09
% Intended LaTeX compiler: pdflatex
\documentclass[11pt]{article}
\usepackage[utf8]{inputenc}
\usepackage[T1]{fontenc}
\usepackage{graphicx}
\usepackage{longtable}
\usepackage{wrapfig}
\usepackage{rotating}
\usepackage[normalem]{ulem}
\usepackage{amsmath}
\usepackage{amssymb}
\usepackage{capt-of}
\usepackage{hyperref}
\usepackage{minted}
\date{\today}
\title{}
\hypersetup{
 pdfauthor={},
 pdftitle={},
 pdfkeywords={},
 pdfsubject={},
 pdfcreator={Emacs 28.2 (Org mode 9.5.5)}, 
 pdflang={English}}
\begin{document}

\tableofcontents


\section{Ensimmäiset 20 artikkelia}
\label{sec:org3a84383}

\subsection{Integrated Development Environments (IDEs)}
\label{sec:orga0c16fc}
(,  2015 )
Ei pääsyä JYU tunnuksilla.

\subsection{Integrated Learning Development Environment for Learning and Teaching C/C++ Language to Novice Programmers}
\label{sec:org47e3d38}
(,  2020 )

\begin{itemize}
\item IDLE pedagogiikkaa ensimmäisen vuode opiskelijoille
\item Monimuotoinen oppimateriaali, formatiivinen palaute, kustomoitu
kääntäjä, visualisointi, nämä yhdistettynä moderneihin pedagogisiin
käytäntöihin
\item Vain C++ -opetukseen, räätälöidyt palautteet kunkin opiskelijan
virheiden, ongelmien ja taitotason perusteella kustomoidun kääntäjän
avulla
\end{itemize}

\subsection{THE ACCEPTANCE OF AN EDUCATIONAL INTEGRATED DEVELOPMENT ENVIRONMENT TO LEARN PROGRAMMING FUNDAMENTALS}
\label{sec:org3b1d145}
(Noor, Nor Farahwahida Mohd and Saad, Aslina and Ibrahim, Abu Bakar and Noor, Norashady Mohd, 2023)

\begin{itemize}
\item C-ohjelmoinnin opetuksee kehitettiin sovellus, C-SOLVIS IDE
\item Web pohjainen sovellus, johon sovellettu erillistä menetelmää
(SUS) käytettävyyden arviointiin
\item väitetään, että tätä voisi soveltaa jatkossa ohjeena vastaaville
opetuksellisen näkökulman IDE ohjelmistoille
\end{itemize}

\subsection{An Overview of Tools for Collecting Data on Software Development and Debugging Processes from Integrated Development Environments}
\label{sec:org1ba6867}
(Zhevaho, O. O., 2021)
\begin{itemize}
\item Kirjallisuuskatsaus, tarkastelee erilaisia työkaluja joilla
seurataan kehittäjän toimintaa IDEssä
\item Väitetään, että artikkeli voi helpottaa ko. työkalun kehittämisessä
\item Löytyi kymmeniä IDE lisäosia joilla voi tarkkailla kehittäjän
käyttäytymistä IDEssä ja löytyi relevanttia kirjallisuutta
\end{itemize}


\subsection{Towards Embedding a Tutoring Companion in the Eclipse Integrated Development Environment}
\begin{itemize}
\item Kirjan osa, jossa tutkitaan avustin lisäosan käyttöä kandiopiskelijoille IDEssä.
\item Lisäosa kerää tietoa kääntämisen, koodaamisen ja ajon aikana koodin oikeellisuudesta, ajankäytöstä, kopioidun tai generoidun koodin määrästä, tuottaa välittömän palautteen tästä
\item Väitetään että tästä voi tulla kätevä työkalu oppimisen prosessissa
\end{itemize}


\subsection{Online Integrated Development Environment (IDE) in Supporting Computer Programming Learning Process during COVID-19 Pandemic: A Comparative Analysis}
\begin{itemize}
\item Web IDE vertailu
\item Aika vähän on hyvinsoveltuvia Web IDEjä, joko liian vähän tukea
  kirjastoille tai sitten vaatii paljon datansiirtokykyä
\end{itemize}

\subsection{Experience report: evolution of a web-integrated software development and verification environment}
\begin{itemize}
\item Web IDE raportti kehitystyöstä
\item Työn tarkoitus oli tuottaa modulaarista/komponenttipohjaista
  kehitystä tukeva IDE
\item IDEä on käytetty kandiopiskelijoiden kursseilla modulaarisen
  kehityksen opettamiseen
\end{itemize}

\subsection{How ``Friendly'' Integrated Development Environments Are?}
\begin{itemize}
\item Oletuksena on, että IDEn käytön opiskeluun menee viikkoja aikaa
\item Käytettävyyskysely useammasta eri IDEstä kandiopiskelijoille
\item Huomiona mm. koodarien tarve olla yhteydessä muihin koodareihin
  IDEn kautta
\end{itemize}

\subsection{A review on effective approach to teaching computer programming to undergraduates in developing countries}
\begin{itemize}
\item Kirjallisuuskatsaus kehittyvien maiden ohjelmoinnin opetukseen
\item Ratkaisuna ehdotetaan Mobile IDEn käyttöä ja suositellaan
  vaihtoehtoja 
\end{itemize}

\subsection{Tools and Techniques for Teaching Computer Programming: A Review}
\begin{itemize}
\item Robottiohjelmointia, pelipohjaista oppimista, pariohjelmointia ja Scratchia väläytellään
  keinoksi opiskeljoiden sitouttamiseen
\item Mm. pariohjelmoinnissa tai yhteisohjelmoinnissa on havaittu, että
  jotkut ohjaajat neuvovat käyttämään IDEä jossa on mahdollisuus
  kehittää yhtäaikaisesti samaa koodia muiden kanssa
\end{itemize}

\subsection{Teaching Students to Fix Programming Errors with Tutorials Embedded in an IDE}
\begin{itemize}
\item Ideana yhdistää IDE käyttöä tms. ohjaavat tutoriaalit suoraan
  IDEen
\item Meneillään oleva sekä tuleva työ toteuttaa tuon käytännössä (ei annettu aikataulua)
\end{itemize}

\subsection{Survey on Adverse Effect of Sophisticated Integrated Development Environments on Beginning Programmers' Skillfulness}
\begin{itemize}
\item Mielipidekysely IDE käytön haasteista (kandiopiskelijoille)
\item Itse iden käyttö vaikuttaa osoittautuneen opiskelijoille hankalaksi
\end{itemize}

\subsection{Principles of Creating an Integrated Development Environment for Educational Computer Systems}
\begin{itemize}
\item Ei pääsyä artikkeliin
\item Mm. didaktisesta näkökulmasta pohdittuna IDE kehitystä
  koulutustarpeisiin
\end{itemize}

\subsection{Comparison of Feedback Strategies for Supporting Programming Learning in Integrated Development Environments (IDEs)}
\begin{itemize}
\item Ei pääsyä artikkeliin
\item Tutkitaan auttaako IDEn kautta saatu palaute onglemien
  ratkomisessa
\item Selvitetty parilla prototyyppi pluginilla Eclipsessä
\item Osallistujien palautteen perusteella suora feedbackki IDEssä oli
  hyödyllinen
\end{itemize}

\subsection{WIDE an interactive Web integrated development environment to practice C programming in distance education}
\begin{itemize}
\item Kehitettiin Web pohjainen IDE C kielen opiskeluun
\item Väitetään että kaikki käyttäjät tutkimusjaksolla sanoivat WIDEn
  helpottavan C ohjelmien kääntämistä ja ajamista työpöytä-IDEen
  verrattuna
\end{itemize}

\subsection{Interactive support for secure programming education}
\begin{itemize}
\item Prototyyppi IDE, joka tuottaa palautetta tietoturvariskeistä ja
  haavoittuvuuksista koodissa
\item Tutkittu reaktioita sisäisiin toimintoihin, dialogeihin tm.
\item Opiskelijat tulivat enemmän tietoiseksi turvallisista
  käytännöistä ja myös toteuttivat niitä
\end{itemize}

\subsection{JavaWIDE (Wiki Integrated Development Environment: redesigning CS1 distance education labs}
\begin{itemize}
\item Ei doita, ei pääsyä
\item Online IDE, helppokäyttöinen aloittelijoille
\end{itemize}

\subsection{A Web-Based Environment to Improve Teaching and Learning of Computer Programming in Distance Education}
\begin{itemize}
\item Ei ehkä relevantti, vanha online kurssitoteutuksia analysoiva tutkimus
\end{itemize}


\section{Lähteet}
\label{sec:orgb7e844c}
\end{document}
