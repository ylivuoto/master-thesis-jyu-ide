% Created 2024-03-23 la 16:09
% Intended LaTeX compiler: pdflatex
\documentclass[11pt]{article}
\usepackage[utf8]{inputenc}
\usepackage[T1]{fontenc}
\usepackage{graphicx}
\usepackage{longtable}
\usepackage{wrapfig}
\usepackage{rotating}
\usepackage[normalem]{ulem}
\usepackage{amsmath}
\usepackage{amssymb}
\usepackage{capt-of}
\usepackage{hyperref}
\usepackage{minted}
\date{\today}
\title{}
\hypersetup{
 pdfauthor={},
 pdftitle={},
 pdfkeywords={},
 pdfsubject={},
 pdfcreator={Emacs 28.2 (Org mode 9.5.5)}, 
 pdflang={English}}
\begin{document}

\tableofcontents


\section{Ensimmäiset 20 artikkelia}
\label{sec:org3a84383}

\subsection{Integrated Development Environments (IDEs)}
\label{sec:orga0c16fc}
(,  2015 )
Ei pääsyä JYU tunnuksilla.

\subsection{Integrated Learning Development Environment for Learning and Teaching C/C++ Language to Novice Programmers}
\label{sec:org47e3d38}
(,  2020 )

\begin{itemize}
\item IDLE pedagogiikkaa ensimmäisen vuode opiskelijoille
\item Monimuotoinen oppimateriaali, formatiivinen palaute, kustomoitu
kääntäjä, visualisointi, nämä yhdistettynä moderneihin pedagogisiin
käytäntöihin
\item Vain C++ -opetukseen, räätälöidyt palautteet kunkin opiskelijan
virheiden, ongelmien ja taitotason perusteella kustomoidun kääntäjän
avulla
\end{itemize}

\subsection{THE ACCEPTANCE OF AN EDUCATIONAL INTEGRATED DEVELOPMENT ENVIRONMENT TO LEARN PROGRAMMING FUNDAMENTALS}
\label{sec:org3b1d145}
(Noor, Nor Farahwahida Mohd and Saad, Aslina and Ibrahim, Abu Bakar and Noor, Norashady Mohd, 2023)

\begin{itemize}
\item C-ohjelmoinnin opetuksee kehitettiin sovellus, C-SOLVIS IDE
\item Web pohjainen sovellus, johon sovellettu erillistä menetelmää
(SUS) käytettävyyden arviointiin
\item väitetään, että tätä voisi soveltaa jatkossa ohjeena vastaaville
opetuksellisen näkökulman IDE ohjelmistoille
\end{itemize}

\subsection{An Overview of Tools for Collecting Data on Software Development and Debugging Processes from Integrated Development Environments}
\label{sec:org1ba6867}
(Zhevaho, O. O., 2021)
\begin{itemize}
\item Kirjallisuuskatsaus, tarkastelee erilaisia työkaluja joilla
seurataan kehittäjän toimintaa IDEssä
\item Väitetään, että artikkeli voi helpottaa ko. työkalun kehittämisessä
\item Löytyi kymmeniä IDE lisäosia joilla voi tarkkailla kehittäjän
käyttäytymistä IDEssä ja löytyi relevanttia kirjallisuutta
\end{itemize}


\subsection{Towards Embedding a Tutoring Companion in the Eclipse Integrated Development Environment}
\begin{itemize}
\item Kirjan osa, jossa tutkitaan avustin lisäosan käyttöä kandiopiskelijoille IDEssä.
\item Lisäosa kerää tietoa kääntämisen, koodaamisen ja ajon aikana koodin oikeellisuudesta, ajankäytöstä, kopioidun tai generoidun koodin määrästä, tuottaa välittömän palautteen tästä
\item Väitetään että tästä voi tulla kätevä työkalu oppimisen prosessissa
\end{itemize}


\subsection{Online Integrated Development Environment (IDE) in Supporting Computer Programming Learning Process during COVID-19 Pandemic: A Comparative Analysis}
\begin{itemize}
\item Web IDE vertailu
\item Aika vähän on hyvinsoveltuvia Web IDEjä, joko liian vähän tukea
  kirjastoille tai sitten vaatii paljon datansiirtokykyä
\end{itemize}

\subsection{Experience report: evolution of a web-integrated software development and verification environment}
\begin{itemize}
\item Web IDE raportti kehitystyöstä
\item Työn tarkoitus oli tuottaa modulaarista/komponenttipohjaista
  kehitystä tukeva IDE
\item IDEä on käytetty kandiopiskelijoiden kursseilla modulaarisen
  kehityksen opettamiseen
\end{itemize}


\section{Lähteet}
\label{sec:orgb7e844c}
\end{document}
